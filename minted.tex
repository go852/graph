\documentclass{article}
\usepackage{minted}
\usepackage{xcolor}

\begin{document}
% 定义自定义颜色(RGB值)
\definecolor{uLightGray}{rgb}{0.9, 0.9, 0.9}  % 这里是一个淡灰色背景

\setminted{
  frame=single,               % 设置边框类型
  framesep=5pt,              % 设置边框和代码的间距
  linenos=true,                 % 显示行号
  bgcolor=uLightGray,     % 设置背景颜色
  fontsize=\normalsize,    % 设置字体大小
  numbersep=5pt,            % 设置行号和代码的间距
  style=xcode,                  % 选择语法高亮风格
  xleftmargin=24pt,           % 设置左侧缩进
  xrightmargin=24pt,         % 设置右侧缩进 
  fontfamily=tt,                  % 字体:默认tt,等宽字体;rm,衬线字体;sf,无衬线字体
  numberblanklines=false, % 空行是否显示行号
}

\begin{minted}{python}
def greet(name):
    print(f"Hello, {name}!")
    return name
    printf("");

greet("Alice")
\end{minted}

\begin{minted}{c++}
#include <iostream>

int main() {
    std::cout << "Hello, World!" << std::endl; // 输出
    return 0;
}

\end{minted}

\end{document}
